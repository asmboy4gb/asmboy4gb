\documentclass[a5paper,9pt]{extarticle}
\usepackage{graphicx,amssymb, amstext, amsmath, epstopdf, booktabs, verbatim, gensymb, geometry, appendix, natbib}
\geometry{a5paper,total={5.3in, 7.4in}}

\usepackage{multicol} 
\usepackage[ngerman]{babel}
\usepackage[utf8]{inputenc}
\usepackage[T1]{fontenc}

\usepackage[babel,german=guillemets]{csquotes}

\usepackage[sc]{mathpazo}

\usepackage{verbatim} 
\usepackage{listings} 


\usepackage{blindtext}
\usepackage{titlesec}
\usepackage{xcolor}

\NewDocumentCommand\Address{m}{%
\texttt{\colorbox{lightgray!30}{#1}}%
}

\begin{document}
\pagestyle{empty}
\titlespacing{\section}{0pt}{0pt}{0pt}
\titlespacing{\subsection}{0pt}{2pt}{2pt}
\titlespacing{\subsubsection}{0pt}{1pt}{1pt}


\centerline{\fontsize{20}{28}{\bf{asmboy4gb}} C H E A T  S H E E T} % Poster title
\begin{multicols}{2}
\subsection*{Graphics}
All the values below can only be changed while in a VBlank.
\subsubsection*{VRAM tiles}
Every tile (8x8 pixels) consists of 16 bytes. 2 bit per pixel give the color index. Two bytes represent one line. For every pixel the most significant bit is taken from the second byte
 and the least significant bit from first byte for the line.
 \begin{tabular}{ c c c }
    \bf Address & \bf Object & \bf Background \\ 
    \Address{\$8000-\$87FF} & 0-128 & - \\  
    \Address{\$8800-\$8FFF} & 128-255 & 128-255 \\
    \Address{\$9000-\$97FF}  & - & 0-127    
   \end{tabular}
The background is 32x32 tiles big (256x256 pixels). Every byte of the tilemap is one id of background tile from the table above.
\begin{tabular}{ c c }
    \Address{\$9800-\$981F} & first line \\  
    \Address{\$9820-\$983F} & second line \\
     ... & \\
     \Address{\$9BE0-\$9BFF}  & last line   
   \end{tabular}

\subsubsection*{Object attribute memory (OAM)}
   Every of the 40 objects has 4 bytes in the memory area \Address{\$FE00-\$FE9F} \\
\begin{tabular}{ c c }
    Byte 0 & Y position. Top = 16 \\  
    Byte 1 & X position. Left = 8 \\
    Byte 2 & Object VRAM tile number  \\
    Byte 3 & Is a bitmap, see table below \\
   \end{tabular}
\\
\begin{tabular}{ c c }
    Bit 4 & Palette number \\
    Bit 5 & X flip if set \\
    Bit 6 & Y flip if set \\
    Bit 7 & Background before objects if set
   \end{tabular}
\subsubsection*{Moving Background view}
With \Address{\$FF42} the Y position and with \Address{\$FF43} the X position of the visible view is controlled. The visible view is 160x144 pixels out of the 256x256 pixels in the background tilemap.
\subsubsection*{Color palettes}
\Address{\$FF47} Is the color palette for the background tiles. The first two bits give the color for index 0, third and forth bit for color of index 1 etc.
\\
\begin{tabular}{ c c }
    0 & White \\
    1 & Light gray \\
    2 & Darker gray \\
    3 & Black
   \end{tabular}
   \\
\Address{\$FF48} and \Address{\$FF49} give the color for the object tiles. Which of this two palettes is taken is choosen by bit 4 in byte 3 in OAM. They work as the background palette but 0 is transparent instead of white.
\subsubsection*{\textasciitilde 16kHz timer: \Address{\$FF04} }
\subsection*{Functions to call}
All the functions below can be called with their address. They dont modify any other registers besides the argument and the return.
% Keep in sync with asmboy4gb.map
\subsubsection*{\Address{\$0150} Wait for next VBlank}
No arguments, no return. After this returns the VBlank just started and can be fully used to access graphic memory.
\subsubsection*{\Address{\$0163} Clear Background}
No arguments, no return. Sets the background to the empty tile 0 which is set to all white. After this returns the VBlank just started and can be fully used to access graphic memory.
\subsubsection*{\Address{\$018C} Get DPad}
Returns all keys pressed in register a. If the bit is 0 it means the corresponding key is pressed

\begin{tabular}{ c c }
    Bit 0 & Right \\
    Bit 1 & Left\\
    Bit 2 &  Up \\
    Bit 3 & Down
   \end{tabular}

\subsubsection*{\Address{\$019E} Get Buttons}
Same as DPad. Start button will exit.
   \begin{tabular}{ c c }
       Bit 0 & A Button \\
       Bit 1 & B Button\\
       Bit 2 &  Select Button \\
       Bit 3 & Start Button 
      \end{tabular}


\subsubsection*{\Address{\$01B0} Print Number}
Number to print in a. Destination in tilemap \Address{\$9800-\$9BFF} in hl

\subsubsection*{\Address{\$01CC} Print Game Over Screen}
Destination in tilemap \Address{\$9800-\$9BFF} in hl. Returns only after A was pressed.

\end{multicols}
\end{document}